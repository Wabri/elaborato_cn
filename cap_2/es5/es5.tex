Il metodo di bisezione \'e applicabile se la funzione \(f\) \'e
\begin{enumerate}

\item continua nell'intervallo \( [a,b] \)
\item tale che \( f(a)f(b)<0 \)

\end{enumerate}

Dal momento che lo zero delle funzioni \( f_1(x)=(x-\pi)^{10} \) e \( f_2(x)=e^{2x}(x-\pi)^{10} \) risulta essere in \( x=\pi \), ultilizzare come punto iniziale \( x_0 = 5 > \pi \) non porterebbe chiaramente alla determinazione dello zero ancor prima di valutare la regolarit\'a della funzione. 
Analizzando poi le due \( f \) si nota subito che \( 
\neg \exists x | f_2(x)<0, \forall x \in \mathbb{R} 
\).

 

