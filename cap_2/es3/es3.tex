\begin{flushleft}
Per ricercare la radice quadrata di un numero è possibile sfruttare una modifica al problema delle radici di una funzione. Infatti partendo da $x=\sqrt{\alpha}$ è possibile svilupparla trovando una funzione $f(x)$ utilizzabile per il metodo delle secanti:
\[
x = \sqrt{\alpha} 
\]
\[
x^{2} = \big(\sqrt{\alpha}\big)^{2} 
\]
\[
x^{2} - \alpha = 0
\]
possiamo quindi considerare $f(x) = x^{2} - \alpha$. La procedura iterativa è definita quindi da:
\[
x_{n+1} = \frac{(x_n^2-\alpha)\cdot x_{n-1} - (x_{n-1}^2-\alpha)x_n}{x_n^2-x_{n-1}^2} = \frac{\alpha\cdot (x_n-x_{n-1})+x_n\cdot x_{n-1}\cdot (x_n - x_{n-1})}{(x_n+x_{n-1})\cdot (x_n-x_{n-1})} = \frac{\alpha+x_n\cdot x_{n-1}}{x_n+x_{n-1}}
\]
che ci permette di implementare il seguente script matlab e la funzione y = SecSqrt(alpha, x\_0, imax, tol):
\lstinputlisting[language=matlab]{cap_2/es3/es3.m}
I risultati ottenuti dall'utilizzo del metodo delle secanti sono: \newline \\
\scalebox{0.9} {
\begin{tabular}{|c|c|c|c|c|}
\hline
i & metodo di Newton & metodo delle secanti & \big|newton-$\sqrt{3}$\big| & \big|secanti-$\sqrt{3}$\big|\\
\hline
1 & 1.750000000000000e+00 & 1.736842105263158e+00 & 1.794919243112281e-02 & 4.791297694280772e-03 \\
2 & 1.732142857142857e+00 & 1.732142857142857e+00 & 9.204957398001312e-05 & 9.204957397979108e-05 \\
3 & 1.732050810014727e+00 & 1.732050934706042e+00 & 2.445850189047860e-09  & 1.271371643518648e-07 \\
4 &  1.732050807568877e+00& 1.732050807572256e+00 & 0 & 3.378630708539276e-12 \\
5 & --------- & 1.732050807568877e+00 & --------- & 2.220446049250313e-16 \\
\hline
\end{tabular}
} 
\newline \\
Dalla tabella possiamo dunque notare che il metodo di Newton arriva più velocemente al valore cercato rispetto al metodo delle secanti.
\end{flushleft}