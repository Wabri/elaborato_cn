\begin{flushleft}
E' possibile effettuare gli stessi passaggi dell'esercizio precedente, ricordandosi che la radice in questo caso non è quadrata ma ennesima:
\[
x = \sqrt[n]{\alpha} 
\]
\[
x^{n} = \big(\sqrt[n]{\alpha}\big)^{n} 
\]
\[
x^{n} - \alpha = 0
\]
consideriamo quindi la funzione $f(x) = x^{n} - \alpha$ e $f'(x) = n\cdot x^{n-1}$ . La procedura iterativa è definita quindi da:
\[
x_{n+1} = x_{n} - \frac{f(x_{n})}{f'(x_{n})} = x_{n}-\frac{x_{n}^n-\alpha}{n\cdot x_{n}^{n-1}} = x_{n} - \frac{x_{n}}{n} + \frac{\alpha}{n\cdot x_{n}^{n-1}} = 
\]
\[
= \Big((n-1)\cdot x_n- \frac{\alpha}{x_n^{n-1}}\Big) \cdot \frac{1}{n} = \frac{\Big( (n-1)\cdot x_n^{n}+\alpha\Big)}{n\cdot x_n^{n-1}}
\]
La radice da approssimare in questo caso ha grado ennesimo quindi sono necessarie delle modifiche alla funzione matlab usata nell'esercizio precedente che chiamiamo y = NewtonSqrt(n, alpha, x\_0, imax, tol). Lo script MatLab corrispondente ai casi $n = 3,4,5$ è il seguente:
\lstinputlisting[language=Matlab]{cap_2/es2/es2.m}
\lstinputlisting[language=matlab]{cap_2/NewtonSqrtN.m}
Mostriamo l'output in forma tabellare con $i$ che rappresenta le iterazioni del metodo e $x_i$ i relativi risultati:
\begin{center}
\begin{tabular}{|c|c|c|c|}
\hline
i & $x_i$ con $n=3$ & $x_i$ con $n=4$ & $x_i$ con $n=5$\\
\hline
1 & 1.631784138709347e+00 & 1.771797299323380e+00 & 1.943788863498140e+00 \\
2 & 1.463411989089094e+00 & 1.463688102853308e+00 & 1.597060655491283e+00 \\
3 & 1.442554125137959e+00 & 1.336940995805593e+00 & 1.369877122538772e+00 \\
4 & 1.442249634601091e+00 & 1.316557487370408e+00 & 1.266284124539191e+00 \\
5 & 1.442249570307411e+00 & 1.316074279204018e+00 & 1.246387399421677e+00 \\
6 & ------------ & 1.316074012952573e+00 & 1.245731630753065e+00 \\
7 & ------------ & ------------ & 1.245730939616284e+00 \\
\hline
\end{tabular}
\end{center}
\end{flushleft}