Ecco le function da noi utilizzate nel corso del capitolo :

\begin{itemize}

\item Differenze divise :
\lstinputlisting[language=Matlab]{cap_4/differenzeDivise.m}

\item Interpolazione nella forma di Newton
\lstinputlisting[language=Matlab]{cap_4/formaNewton.m}

\item Differenze divise per polinomio di Hermite :
\lstinputlisting[language=Matlab]{cap_4/differenzeDiviseHermite.m}

\item Interpolazione con polinomio di Hermite :
\lstinputlisting[language=Matlab]{cap_4/hermite.m}

\item Ascisse equispaziate
\lstinputlisting[language=Matlab]{cap_4/ascisseEquispaziate.m}

\item Ascisse di Chebyshev
\lstinputlisting[language=Matlab]{cap_4/chebyshev.m}

\item Algoritmo di Horner Generalizzato
\lstinputlisting[language=Matlab]{cap_4/HornerGeneralizzato.m}

\end{itemize}

\subsection{\textbf{Esercizio 4.1}}
\lstinputlisting[language=Matlab]{cap_4/poliNewton.m}

\subsection{\textbf{Esercizio 4.2}}
\begin{flushleft}Il codice MatLab usato è il seguente:
\lstinputlisting[language=matlab]{cap_4/es2/es2.m}

che genera questi risultati:

\[
\begin{tabu}{cccc}
RungeEq & RungeCheb & SinEq & SinCheb \\
\hline
0.646 & 0.4371 & 0.6381 & 0.4371\\
0.4383 & 0.02286 & 0.04127 & 0.02286\\
0.6164 & 0.0004779 & 0.001343 & 0.0004779\\
1.045 & 5.332\cdot 10^{-6} & 2.575\cdot 10^{-5} & 5.332\cdot 10^{-6}\\
1.915 & 3.688\cdot 10^{-8} & 3.238\cdot 10^{-7} & 3.688\cdot 10^{-8}\\
3.612 & 1.734\cdot 10^{-10} & 2.843\cdot 10^{-9} & 1.734\cdot 10^{-10}\\
7.189 & 5.892\cdot 10^{-13} & 1.873\cdot 10^{-11} & 5.892\cdot 10^{-13}\\
14.01 & 3.417\cdot 10^{-15} & 1.261\cdot 10^{-13} & 3.417\cdot 10^{-15}\\
27.51 & 1.776\cdot 10^{-15} & 8.933\cdot 10^{-14} & 1.776\cdot 10^{-15}\\
58.41 & 2.327\cdot 10^{-15} & 1.768\cdot 10^{-13} & 2.327\cdot 10^{-15}
\end{tabu}
\]

Possiamo vedere la differenza tra \ref{RungeEq} e \ref{RungeChe}, nella prima all'aumentare delle ascisse la funzione interpolata degenera, mentre nella seconda già con n=5 abbiamo una buona interpolazione.
Nel caso della seconda funzione possiamo vedere che la differenza tra \ref{SinEq} e \ref{SinChe} non è molto rilevante. Infatti già con 4 ascisse abbiamo un interpolazione quasi perfetta.
Nelle figure \ref{RungeEqErr} \ref{RungeCheErr} \ref{SinEqErr} \ref{SinCheErr} e' possibile vedere l'andamento dell'errore per i vari metodi di interpolazione.
\end{flushleft}
\subsection{\textbf{Esercizio 4.3}}
\lstinputlisting[language=Matlab]{cap_4/momentiSplineNat.m}
\subsection{\textbf{Esercizio 4.4}}
\lstinputlisting[language=Matlab]{cap_4/evaluation.m}
\subsection{\textbf{Esercizio 4.5}}
 Il seguente listato valuta la spline naturale e quella not-a-knot per le funzioni date:

\lstinputlisting[language=matlab]{cap_4/es5/es5.m}

con i risultanti grafici:
grafico runge not-a-knot \ref{runge_nak}
grafico errori runge not-a-knot \ref{erunge_nak}
grafico not-a-knot funzione \ref{sin_nak}
grafico errori not-a-knot funzione \ref{esin_nak}
\subsection{\textbf{Esercizio 4.6}}
\lstinputlisting[language=Matlab]{cap_4/momenti_periodica.m}

\subsection{\textbf{Esercizio 4.7}}
\lstinputlisting[language=Matlab]{cap_2/es7/es7.m}

I dati generati dall'esecuzione del codice sono esposti  nella seguente tabella

\begin{tabular}{l c r}

n & $tol_x$ & y \\
\hline
0 & $10^{-1}$ & 0.500000000000000 \\
7 & $10^{-2}$ & 0.488281250000000 \\
10 & $10^{-3}$ &  0.488769531250000 \\
13 & $10^{-4}$ & 0.488952636718750 \\
16 & $10^{-5}$ & 0.488945007324219 \\
20 & $10^{-6}$ & 0.488943576812744 \\
21 & $10^{-7}$ & 0.488943815231323 \\
26 & $10^{-8}$ & 0.488943792879581 \\
30 & $10^{-9}$ & 0.488943794276565 \\
32 & $10^{-10}$ & 0.488943794392981 \\
\hline
\end{tabular}


\subsection{\textbf{Esercizio 4.8}}
\begin{flushleft}
Generalmente la legge che descrive un dato fenomeno è di tipo polinomiale con un determinato grado $m$:
\[
y = \sum_{k=0}^{m}(a_k\cdot x^k)
\]
Sapendo che $(x_i,y_i)$ sono misure sperimentali è necessario estrapolare i vettore $a_k$ che meglio approssima il polinomio. Per il calcolo di $a_k$ abbiamo scritto la seguente function:
\lstinputlisting[language=matlab]{cap_4/polBetter.m}
Abbiamo poi testato il suo funzionamento con il seguente script:
\lstinputlisting[language=matlab]{cap_4/es8/es8.m}
Che restituisce in output:
\begin{figure}[H]
\includegraphics[left, width=400px, height=200px]{cap_4/es8/es48.png}
\end{figure}
Possiamo notare che il secondo insieme di dati sperimentali non ha un numero di ascisse distinte, infatti $4>m+1=5$ è falso.
\end{flushleft}
\subsection{\textbf{Esercizio 4.9}}
\begin{flushleft}
Lo script che abbimo implementato è il seguente:
\lstinputlisting[language=Matlab]{cap_4/es9/es9.m}
Nello script abbiamo prima definito le funzioni da studiare e creato un vettore rappresentante la costante $\epsilon$ in cui il primo elemento è il caso $\epsilon=0.1$ e il secondo è $\epsilon=0.2$ Dunque nei due cicli annidati vengono calcolati gli elementi delle matrici $y_{i,1}$ e $y_{i,2}$ in cui nella prima colonna si hanno i valori della prima funzione con $\epsilon_1$ e nella seconda con $\epsilon_2$ (il valore di $\gamma_i$ varia per ogni iterazione del ciclo in modo aleatorio). I quali vengono dati in input alla nostra funzione polBetter per effettuare i test richiesti, da cui si ricavano i coefficienti. Alla fine i risultati stampati sono:
\begin{figure}[H]
\includegraphics[left, width=400px, height=200px]{cap_4/es9/es49.png}
\end{figure}
\end{flushleft}
\subsection{\textbf{Esercizio 4.10}}
\lstinputlisting[language=Matlab]{cap_4/es10/es10.m}
vedi \ref{fitting} per il grafico
\begin{figure}
\centering
\includegraphics[width=\textwidth]{cap_4/es10/untitled}
\caption{Confronto dei due grafici dell'esercizio 4.8}
\label{fitting}
\end{figure}
