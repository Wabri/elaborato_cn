\begin{flushleft}Il codice MatLab usato è il seguente:
\lstinputlisting[language=matlab]{cap_4/es2/es2.m}

che genera questi risultati:

\[
\begin{tabu}{cccc}
RungeEq & RungeCheb & SinEq & SinCheb \\
\hline
0.646 & 0.4371 & 0.6381 & 0.4371\\
0.4383 & 0.02286 & 0.04127 & 0.02286\\
0.6164 & 0.0004779 & 0.001343 & 0.0004779\\
1.045 & 5.332\cdot 10^{-6} & 2.575\cdot 10^{-5} & 5.332\cdot 10^{-6}\\
1.915 & 3.688\cdot 10^{-8} & 3.238\cdot 10^{-7} & 3.688\cdot 10^{-8}\\
3.612 & 1.734\cdot 10^{-10} & 2.843\cdot 10^{-9} & 1.734\cdot 10^{-10}\\
7.189 & 5.892\cdot 10^{-13} & 1.873\cdot 10^{-11} & 5.892\cdot 10^{-13}\\
14.01 & 3.417\cdot 10^{-15} & 1.261\cdot 10^{-13} & 3.417\cdot 10^{-15}\\
27.51 & 1.776\cdot 10^{-15} & 8.933\cdot 10^{-14} & 1.776\cdot 10^{-15}\\
58.41 & 2.327\cdot 10^{-15} & 1.768\cdot 10^{-13} & 2.327\cdot 10^{-15}
\end{tabu}
\]

Possiamo vedere la differenza tra \ref{RungeEq} e \ref{RungeChe}, nella prima all'aumentare delle ascisse la funzione interpolata degenera, mentre nella seconda già con n=5 abbiamo una buona interpolazione.
Nel caso della seconda funzione possiamo vedere che la differenza tra \ref{SinEq} e \ref{SinChe} non è molto rilevante. Infatti già con 4 ascisse abbiamo un interpolazione quasi perfetta.
Nelle figure \ref{RungeEqErr} \ref{RungeCheErr} \ref{SinEqErr} \ref{SinCheErr} e' possibile vedere l'andamento dell'errore per i vari metodi di interpolazione.
\end{flushleft}