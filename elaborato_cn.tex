\documentclass[a4paper]{article}

%\usepackage[utf8]{inputenc}

\usepackage{hyperref}
\usepackage{xcolor}
\usepackage{graphicx}
\usepackage[english]{babel}
\usepackage[T1]{fontenc}
\usepackage{url}
\usepackage{import}
\usepackage{color}
\usepackage{fancyhdr}
\usepackage{amssymb}
\usepackage{mathtools}
\usepackage[margin=2.5cm]{geometry} 
\usepackage{listings}
\usepackage[utf8x]{inputenc}
\usepackage[numbered,framed]{matlab-prettifier}

\let\ph\mlplaceholder % shorter macro
\lstMakeShortInline"

\lstset{
  style              = Matlab-editor,
  basicstyle         = \mlttfamily,
  escapechar         = ",
  mlshowsectionrules = true,
}


\begin{document}
 

\title{Elaborato di \\ \textbf{Calcolo Numerico}}

\author{Giovanni \emph{Bindi} - \texttt{5530804} - \href{mailto:giovanni.bindi@stud.unifi.it}{\textit{giovanni.bindi@stud.unifi.it}}
   \and Gabriele \emph{Gemmi} - \texttt{5602433} -
   \href{mailto:gabriele.gemmi@stud.unifi.it}{\textit{gabriele.gemmi@stud.unifi.it}}
   \and Gabriele \emph{Puliti} - \texttt{5300140} - \href{mailto:gabriele.puliti@stud.unifi.it}{\textit{gabriele.puliti@stud.unifi.it}}} 


\maketitle

\tableofcontents


\newpage
\section{\textbf{Capitolo 1}}

\subsection{\textbf{Esercizio 1.1}}
Per definizione di metodo iterativo convergente si ha che 
\[
	\lim_{k \to +\infty}\ x_k = x^* 
\]
Supponendo la funzione \( \Phi(x_n) \) uniformemente continua vale
\[ 
	\lim_{k \to +\infty}\ \Phi(x_k) = x^*  = \Phi ( \lim_{k \to +\infty}\ x_k ) = x^* 
\]
Per definizione \'e \( \Phi(x_n) = \ x_{k+1} \) e quindi
\[
	\lim_{k \to +\infty}\ \Phi(x_k) = \lim_{k \to +\infty} \ x_{k+1} = x^*	
\]
Da cui otteniamo che \( x^* \) e'\ un punto fisso per la funzione \( \Phi(x_n) \) , ovvero che \( x^* = \Phi(x^*) \).

\subsection{\textbf{Esercizio 1.2}}

Dal momento che le variabili intere di 2 byte in Fortran vengono gestite in Modulo e Segno, la variabile \texttt{n}, inizializzata con

\begin{lstlisting}[language=Fortran]
integer*2 n
\end{lstlisting}

varia tra \( - 2^{15} \leq n \leq 2^{15} - 1 \) e quindi tra  \( -32768 \leq n \leq 32767 \). \\
Andando quindi ad eseguire la somma \( (32767+1)_{10} = (0111111111111111 + 1)_{2,MS} = (11111111111111111)_{2,MS} = (-327628)_{10} \)

\subsection{\textbf{Esercizio 1.3}}

Per definizione si ha che la precisione di macchina \(u\) per arrotondamento e'\ data da
\(
u=\frac{1}{2} b ^{1-m}
\). \\
Se \(b=8, m=5\) si ha \( u = \frac{1}{2}\cdot 8^{-4} = 1,2207031 \cdot 10^{-4} \)

\subsection{\textbf{Esercizio 1.4}}

\subsection{\textbf{Esercizio 1.5}}

\subsection{\textbf{Esercizio 1.6}}
Codice dell'esercizio 6 : 
\lstinputlisting[language=Matlab]{cap_1/es6.m}
\subsection{\textbf{Esercizio 1.7}}

\subsection{\textbf{Esercizio 1.8}}

\subsection{\textbf{Esercizio 1.9}}

\subsection{\textbf{Esercizio 1.10}}

\subsection{\textbf{Esercizio 1.11}}

\subsection{\textbf{Esercizio 1.12}}

\section{\textbf{Capitolo 2}}



\end{document}