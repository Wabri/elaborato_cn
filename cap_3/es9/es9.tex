Si ha ovviamente che
\[
A = \frac{1}{2}(A+A^T) + \frac{1}{2}(A-A^T)=\frac{1}{2}A + \frac{1}{2}A^T + \frac{1}{2}A - \frac{1}{2}A^T
\]
\\
Definendo $A_s \doteq \frac{1}{2}(A+A^T)$ si mostra come $A_s = A_s^T$ infatti
\[
\frac{1}{2}(A+A^T) = [ \frac{1}{2}(A+A^T)]^T = \frac{1}{2}(A+A^T)^T = \frac{1}{2}(A^T+(A^T)^T) = \frac{1}{2}(A+A^T)^T.
\]
\\
Da cui $A_s$ \'e detta parte \textbf{simmetrica} di $A$.
\\
Definendo $A_a \doteq \frac{1}{2}(A-A^T)$ si mostra come $A_a = -A_a^T$ infatti
\[
\frac{1}{2}(A-A^T) = -\frac{1}{2}(A-A^T)^T = -\frac{1}{2}(A^T-(A^T)^T) = -\frac{1}{2}(A^T-A) = \frac{1}{2}(A-A^T).
\]
Da cui $A_a$ \'e detta parte \textbf{antisimmetrica} di $A$.
\\
Dato poi un generico vettore $\mathbf{x} \in \mathbf{R}^n$ si ha che $\mathbf{x}^TA\mathbf{x} = \mathbf{x}^TA_s\mathbf{x}$ infatti
\[
\mathbf{x}^TA\mathbf{x} = \mathbf{x}^T(A_s + A_a)\mathbf{x} = \mathbf{x}^TA_s\mathbf{x} + \mathbf{x}^TA_a\mathbf{x} = \frac{1}{2}(\mathbf{x}^TA\mathbf{x} + \mathbf{x}^TA^T\mathbf{x}) + \frac{1}{2}(\mathbf{x}^TA\mathbf{x} - \mathbf{x}^TA^T\mathbf{x}).
\]
Analizzando l'ultimo termine $\frac{1}{2}(\mathbf{x}^TA\mathbf{x} - \mathbf{x}^TA^T\mathbf{x})$ si nota come
\[ \frac{1}{2}(\mathbf{x}^TA\mathbf{x} - \mathbf{x}^TA^T\mathbf{x}) = \frac{1}{2}(\mathbf{x}^TA\mathbf{x} - (A\mathbf{x})^T\mathbf{x})=0
\]
\\
dal momento che, definendo $A\mathbf{x}=\mathbf{y}$ si ha $\mathbf{x}^T\mathbf{y}= \mathbf{y}^T\mathbf{x}.$
Da cui la tesi.  $\qedsymbol$
