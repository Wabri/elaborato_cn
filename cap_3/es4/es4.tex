\begin{flushleft}
L'eliminazione nella prima colonna richiede $n$ somme ed $n$ prodotti per $n-1$ righe, quindi in totale $(n+n)(n-1) = 2n(n-1)$ \texttt{flops}. L'eliminazione della seconda richiede $n-1$ somme ed $n-1$ prodotti per $n-2$ righe, quindi in totale $[(n-1)+(n-1)](n-2) = 2(n-1)(n-2)$ \texttt{flops}. Procedendo la successione fino alla prima riga si ottiene la sommatoria:
\[
\sum_{i=0}^{n} 2(n-i)(n-i+1).
\]
Operando la sostituzione $j = n-i+1$ si ha che la somma diviene :
\[
2 \cdot\sum_{j=n+1}^{1} j(j-1) = 2\cdot\Big( \sum_{j=1}^{n+1}(j^2-j)\Big) = 2\cdot\Big( \sum_{j=1}^{n+1}(j^2) - \sum_{j=1}^{n+1}(j)\Big) = 2\cdot\Big(\sum_{j=1}^{n-1}(j^2) + n^2 + (n+1)^2 - \sum_{j=1}^{n-1}(j) - n - n - 1 \Big) =
\]
\[
= 2\cdot \Big(\frac{n\cdot(n-1)\cdot(2n-1)}{6}+2\cdot n^2+ 2\cdot n+1 - \frac{n\cdot (n-1)}{2} - 2\cdot n -1 \Big) = 2 \cdot \Big( \frac{2\cdot n^3 - 3\cdot n^2 +n}{6} +2\cdot n^2 - \frac{n^2-n}{2} \Big) =
\]
\[
=2 \cdot \Big( \frac{n^3}{3} - \frac{n^2}{2} + \frac{n}{6} +2\cdot n^2 - \frac{n^2}{2} + \frac{n}{2}\Big) = 2\cdot \Big(\frac{n^3}{3} + n^2 + \frac{2}{3} \cdot n \Big) \approx \frac{2}{3} \cdot n^3
\]

quindi si ha che il numero di flop è circa $\frac{2}{3}\cdot n^3$, come volevamo dimostrare.
\end{flushleft}