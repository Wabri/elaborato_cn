L'eliminazione nella prima colonna richiede $n$ somme ed $n$ prodotti per $n-1$ righe, quindi in totale $(n+n)(n-1) = 2n(n-1)$ \texttt{flops}. L'eliminazione della seconda richiede $n-1$ somme ed $n-1$ prodotti per $n-2$ righe, quindi in totale $[(n-1)+(n-1)](n-2) = 2(n-1)(n-2)$ \texttt{flops}.\\
Procedendo per induzione si ha che il numero totale di operazioni \'e
\[
\sum_{i=0}^{n} 2(n-i)(n-i+1). \hspace{50pt} (1)
\]
Operando la sostituzione $y \doteq n-i+1$ si ha che la (1) diviene :
 
\[
2 \sum_{j=0}^{n-1} j(j-1) = 2 \sum_{j=0}^{n-1}j^2 + j = 2 ( \frac{1}{3}n^3 - \frac{1}{3}n)
\]
Asintoticamente quindi proprio $\frac{2}{3}n^3$ \texttt{flops}.