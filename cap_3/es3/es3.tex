Sia $A \in M_{n \times n}$ una matrice triangolare superiore non singolare.\\
Se $A$ \'e invertibile allora pu\'o essere scritta come $A=D(\mathit{I_n}+U)$ dove $D$ \'e una matrice diagonale per cui vale $diag(D)=diag(A)$, $\mathit{I_n}$ \'e la matrice identit\'a ed $U$ \'e una matrice triangolare strettamente superiore ($diag(U)=0$).\\
Per le propriet\'a delle matrici triangolari strettamente superiori si ha che $U^n = 0$ (ovvero che la matrice $U$ \'e nilpotente), mentre per le propriet\'a delle matrici diagonali si ha che $D^{-1}$ \'e ancora una matrice diagonale.
\\
Volendo provare che $A^{-1}=(\mathit{I_n}+U)^{-1}D^{-1}$ espandiamo in serie $ (\mathit{I_n}+U)^{-1}$:
\[
(\mathit{I_n}+U)^{-1}=\mathit{I_n}-U+U^2-...+(-1)^{n-1}U^{n-1}. (1)
\]
Essendo questa una serie di somme e prodotti di matrici triangolari superiori la matrice inversa $A^{-1}$ sar\'a in generale una matrice triangolare superiore.
Nel caso la matrice $A$ sia a diagonale unitaria anche la sua inversa avr\'a diagonale unitaria, dal momento che $D = I_n$, la cui inversa \'e ancora $I_n$.
