Sia $\mathbf{v} \in \mathbb{R}^n| \mathbf{v} \neq \mathbf{0}$ e $A \in M_{n \times n}$.
$A$ si dice sdp (simmetrica definita positiva) se \'e simmetrica ($A$=$A^T$) e se $\mathbf{v}^TA\mathbf{v} > 0.$
Equivalentemente una matrice si dice sdp se \'e simmetrica ed i suoi autovalori sono $> 0.$
Inoltre una matrice $B \in M_{n \times n}$ si dice nonsingolare se $\det{B} \neq 0.$
\\
Dal momento che il polinomio caratteristico \'e invariante per similitudine le matrici quadrate $A$ ed $A^T$ hanno gli stessi autovalori.
Inoltre le matrici $A^TA$ ed $AA^T$ sono simmetriche dal momento che:
\[
(AA^T)^T = (A^T)^TA^T = AA^T ,\]
\[
(A^TA)^T = A^T(A^T)^T = A^TA.
\]
Vale poi $\det{(A^TA)} = \det{(AA^T)} = \det{A}\det{A^T} = (\det{A})^2.$
\\
Dimostriamo quindi che $AA^T$ e $A^TA$ sono definite positive:
\[
\mathbf{v}^TAA^T\mathbf{v} = (A^T\mathbf{v})^T(A^T\mathbf{v}) > 0, \]
\[
\mathbf{v}^TA^TA\mathbf{v} = (A\mathbf{v})^T(A\mathbf{v}) > 0. \qedsymbol
\]
