Sia $\mathbf{v} \in \mathbb{R}^n| \mathbf{v} \neq \mathbf{0}$ e $A \in M_{n \times n}$.
$A$ si dice sdp (simmetrica definita positiva) se \'e simmetrica ($A$=$A^T$) e se $\mathbf{v}^TA\mathbf{v} > 0.$
Equivalentemente una matrice si dice sdp se \'e simmetrica ed i suoi autovalori sono $> 0.$
Inoltre una matrice $B \in M_{n \times n}$ si dice nonsingolare se $\det{B} \neq 0.$
\\
Dal momento che il polinomio caratteristico \'e invariante per similitudine le matrici quadrate $A$ ed $A^T$ hanno gli stessi autovalori.
Inoltre le matrici $A^TA$ ed $AA^T$ sono simmetriche.
Vale poi $\det{(A^TA)} = \det{(AA^T)} = \det{A}\det{A^T} = (\det{A})^2.$

finisco stase
