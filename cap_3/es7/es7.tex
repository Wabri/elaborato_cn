\begin{flushleft}
Per essere SDP una matrice $A \in \mathbb{R}^{n \times n}$ deve sottostare a due proprietà:
\begin{itemize}
    \item
    deve essere simmetrica, cioè $A=A^T$;
    \item 
    $\forall x \in \mathbb{R}^n$ tale che $x \neq 0$ vale $x^TAx>0$
\end{itemize}
La matrice $A$ essendo non singolare è anche SDP.
Le matrici $AA^T$ e $A^TA$ per essere SDP devono dimostrare le proprietà sopra:
\begin{itemize}
    \item simmetriche: 
    \[
    (AA^T)^T = (A^T)^TA^T = AA^T
    \]
    \[
    (A^TA)^T = A^T(A^T)^T = A^TA.
    \]
    \item definite positive:
    \[
    x^TAA^Tx = xx^TAA^Txx^T = x(A^Tx)^T(x^TA)^Tx^T = (\underbrace{x^TA^Tx}_{>0})^T \cdot (\underbrace{x^TAx}_{>0})^T > 0
    \]
    \[
    x^TA^TAx = xx^TA^TAxx^T = x(Ax)^T(x^TA^T)^Tx^T = (\underbrace{x^TAx}_{>0})^T \cdot (\underbrace{x^TA^Tx}_{>0})^T > 0
    \]
\end{itemize}
possiamo quindi affemrare che le matrici $AA^T$ e $A^TA$ sono SDP.
\end{flushleft}