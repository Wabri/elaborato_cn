\begin{flushleft}
Per dimostrare che la matrice $A \in \mathbb{R}^{n \times n}$ sia SDP deve sottostare a due proprietà:
\begin{itemize}
    \item
    deve essere simmetrica, cioè $A=A^T$;
    \item 
    $\forall x \in \mathbb{R}^n$ tale che $x \neq 0$ vale $x^TAx>0$
\end{itemize}
Le matrici $AA^T$ e $A^TA$ per essere SDP devono dimostrare le proprietà sopra:
\begin{itemize}
    \item proprietà di simmetria:
    \[
    (AA^T)^T = (A^T)^TA^T = AA^T
    \]
    \[
    (A^TA)^T = A^T(A^T)^T = A^TA.
    \]
    la proprietà è quindi confermata;
    \item definita positiva:
    \[
    x^TAA^Tx = xx^TAA^Txx^T = x(A^Tx)^T(x^TA)^Tx^T = (\underbrace{x^TA^Tx}_{>0})^T \cdot (\underbrace{x^TAx}_{>0})^T > 0
    \]
    \[
    x^TA^TAx = xx^TA^TAxx^T = x(Ax)^T(x^TA^T)^Tx^T = (\underbrace{x^TAx}_{>0})^T \cdot (\underbrace{x^TA^Tx}_{>0})^T > 0
    \]
    anche questa proprietà è confermata.
\end{itemize}
Tenendo conto che la matrice $A$ è non singolare se le sue righe sono linearmente indipendenti allora per ogni vettore x non nullo la combinazione lineare $\sum_{i=1}^{n} (x_i\cdot A_i)$ deve essere un vettore non nullo e vale anche l'inverso, possiamo quindi affermare che le matrici $AA^T$ e $A^TA$ sono simmetriche definite positive.
\end{flushleft}