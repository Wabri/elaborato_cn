\textit{Esempio per esercizio \textbf{3.5} e \textbf{3.6}}:\\ \\
Sia 
\( A = 
\begin{pmatrix}
  0 & -3 & 8 \\
  -1 & 8 & 7 \\
  1 & 3 & 0
 \end{pmatrix}
\) la matrice da noi scelta per la risoluzione dell'esercizio. \\ \\ Fattorizziamola con il metodo LU a pivoting parziale, da noi denominato \texttt{LUP(A)}, avendo cura di riscrivere il risultato nella stessa locazione di memoria. \\ \\ Adesso quindi $A = 
\begin{pmatrix}
  -1.0000 & 8.0000 & 7.0000 \\
  -1.0000 & 11.0000 & 7.0000 \\
  0 & -0.2727 & 9.9091
 \end{pmatrix}
$ e $\mathbf{p} = (2,3,1)$ vettore di permutazione.
\\ \\ 
Dal momento che vale $PA=LU$ si ha che $PA\mathbf{x} = P\mathbf{b} = LU\mathbf{x}$
se adesso scegliamo $\mathbf{b} = 3.1416,1.1618,2.7183)^T$, utilizzando la funzione \texttt{solveLinearLUP(A,p,b)} otteniamo 
\[
PA\mathbf{x} = P\mathbf{b} = \]
\[
\begin{pmatrix}
  -1.0000 & 8.0000 & 7.0000 \\
  -1.0000 & 11.0000 & 7.0000 \\
  0 & -0.2727 & 9.9091
 \end{pmatrix} \begin{pmatrix}x_1\\x_2\\x_3\end{pmatrix} = \begin{pmatrix}
 0 & 1 & 0 \\
  0 & 0 & 1 \\
  1 & 0 & 0
 \end{pmatrix} \begin{pmatrix}3.1416\\1.1618\\2.7183\end{pmatrix}
\]
Da cui 
\[
\mathbf{x} = \begin{pmatrix}2.4692 \\ 0.0830 \\ 0.4238 \end{pmatrix}.
\]\\ 
\textit{Esempio per esercizio \textbf{3.11} e \textbf{3.12}}: \\ \\ 
Sia
\( A = 
\begin{pmatrix}
  14 & 5 & 2 \\
  -5 & 8 & 1\\
  2 & 1 & 4
 \end{pmatrix}
\) la matrice da noi scelta per la risoluzione dell'esercizio. \\ \\ Fattorizziamola con il metodo $LDL^T$ da noi denominato \texttt{fattorizzaLDLt}. La matrice $A$ riscritta con l'informazione del metodo risulta quindi essere \\ \\ \( A = 
\begin{pmatrix}
  14.0000 & 5.0000 & 2.0000 \\
  -0.3571 & 6.2143 & 1.0000\\
  0.1429 & 0.2759 & 3.2414
 \end{pmatrix}
\). \\
Utilizzando lo stesso vettore dei termini noti $\mathbf{b}$ risolviamo quindi il sistema lineare $A\mathbf{x} = \mathbf{p}$ con la function \texttt{solveLinearLDL} : 
\[
\begin{pmatrix}
  14.0000 & 5.0000 & 2.0000 \\
  -0.3571 & 6.2143 & 1.0000\\
  0.1429 & 0.2759 & 3.2414
 \end{pmatrix} \begin{pmatrix}x_1\\x_2\\x_3\end{pmatrix} = \begin{pmatrix}3.1416\\1.1618\\2.7183\end{pmatrix}
\] \\ 
Da cui 
\[
\mathbf{x} = \begin{pmatrix}0.2336\\0.2280\\0.5058\end{pmatrix}
\]

\textbf{ATTENZIONE : NON TORNANO I CONTI....} check it here \url{https://www.wolframalpha.com/input/?i=[[14,5,2],[-0.3571,6.2143,1],[0.1429,0.2759,3.2414]]*[x1,x2,x3]+%3D+[3.1416,1.1618,2.7183]} and here \url{https://www.wolframalpha.com/input/?i=[[-1,8,7],[-1,11,7],[0,-0.2727,9.9091]]*[x1,x2,x3]+%3D+[[0,1,0],[0,0,1],[1,0,0]]*[3.1416,1.1618,2.7183]} \\ codice :

\lstinputlisting[language=Matlab]{cap_3/es14/es14.m}