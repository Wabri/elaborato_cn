\textit{Esempio per esercizio \textbf{3.5} e \textbf{3.6}}:\\ \\
Sia A la matrice da noi scelta per la risoluzione dell'esercizio.
\[ A =
\begin{pmatrix}
  0 & -3 & 8 \\
  -1 & 8 & 7 \\
  1 & 3 & 0
 \end{pmatrix}
\]
Sia b il vettore dei termini noti.
\[{b} = (3.1416, 1.1618, 2.7183)^T\]
Utilizzando i metodi numerici allegati e risolvendo il sistema lineare \[Ax = b\] otteniamo il vettore delle incognite
\[x = ( 2.469228440366973,  0.083023853211009,  0.423833944954129)^T\]
Quindi il vettore residuo \[ r = Ax - b = 1.0*10^{-15}( 0.888178419700125,  0,  0)\]


\textit{Esempio per esercizio \textbf{3.11} e \textbf{3.12}}: \\ \\
Sia A la matrice da noi scelta per la risoluzione dell'esercizio.
\[ A =
\begin{pmatrix}
  14 & 5 & 2 \\
  5 & 8 & 1\\
  2 & 1 & 4
 \end{pmatrix}
\]
Sia b il vettore dei termini noti.
\[{b} = (3.1416, 1.1618, 2.7183)^T\]
Utilizzando i metodi numerici allegati e risolvendo il sistema lineare \[Ax = b\] otteniamo il vettore delle incognite

\[x = ( 0.144645652173913,  -0.021765217391304,  0.612693478260870)^T\]

Quindi il vettore residuo \[ r = Ax - b = 1.0*10^{-15}( 0, -0.444089209850063,  0)\]



\lstinputlisting[language=Matlab]{cap_3/es14/es14.m}
