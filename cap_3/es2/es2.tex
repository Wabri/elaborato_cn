\begin{flushleft}
Una matrice triangolare inferiore $L \in M_{n \times n}$ è detta a diagonale unitaria se i suoi elementi sulla diagonale sono pari a 1:
\[
l_{i,i}=1 \hspace{15pt} \forall i \in [1,..,n] 
\]
Prendiamo una seconda matrice $K \in M_{n \times n}$ triangolare inferiore a diagonale unitaria, calcoliamo il prodotto tra $K$ e $L$:
\[
\sum_{m=1}^n (l_{i,m} \cdot k_{m,j}) = \underbrace{ \sum_{i<j} (l_{i,m} \cdot k_{m,j}) }_{0} + \underbrace{ \sum_{i=j} (l_{i,m} \cdot k_{m,i}) }_{1} + \sum_{i>j} (l_{i,m}\cdot k_{m,j})
\]
La risultante matrice assume valori:
\begin{itemize}
\item $\sum_{i<j} (l_{i,m} \cdot k_{m,j}) = 0 \hspace{15pt} \forall i,j \in [1,..,n] $ 
\item $\sum_{i=j} (l_{i,m} \cdot k_{m,j}) = 1 \hspace{15pt} \forall i,j \in [1,..,n] $ 
\item $\sum_{i>j} (l_{i,m} \cdot k_{m,j}) \in  \mathbb{R} \hspace{15pt} \forall i,j \in [1,..,n] $ 
\end{itemize}
che non è altro che la definizione di matrice triangolare inferiore a diagonale unitaria, come volevamo dimostrare.\\

(Allo stesso modo si può dimostrare per matrici triangolari superiori).
\end{flushleft}