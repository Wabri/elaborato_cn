\section{Capitolo 3}
\underline{Le funzioni usate nei codici seguenti sono in fondo al capitolo}
\subsection{Esercizio 1}
Una matrice \( L \in M_{n \times n} \) si dice triangolare inferiore se \( l_{i,j}=0, \forall i<j \) con \(i,j = 1...n \) ed \( l_{i,j} \in L \).
\\
Date due matrici triangolari inferiori \( L,K \in M_{n \times n} \) la loro somma sar\'a di nuovo una matrice traingolare inferiore \( N \in M_{n \times n} \):
\[
 \forall i<j, \hspace{5pt} l_{i,j} + k_{i,j} = 0 + 0 = 0. 
\]
\\
Una matrice \( U \in M_{n \times n} \) si dice triangolare superiore se \( u_{i,j}=0, \forall i>j \) con \(i,j = 1...n \) ed \( u_{i,j} \in U \).
Date due matrici triangolari superiori  \( U,W \in M_{n \times n} \)  il loro prodotto  \( Z \in M_{n \times n} \):
\[
z_{i,j} = \sum_{k=1}^n u_{i,k} w_{k,j}, \hspace{35pt} \forall i,j \in [1,..,n].
\]
sar\'a di nuovo una matrice triangolare superiore, dal momento che 
\[
z_{i,j} =  \sum_{k=1}^n u_{i,k} w_{k,j} =  \underbrace{ \sum_{k=1}^{i-1} u_{i,k} w_{k,j} }_\text{=0 per k<i}  + \sum_{k=i}^n u_{i,k} w_{k,j} = \sum_{k=i}^n u_{i,k} w_{k,j}.
\]
Dal momento che in quest'ultimo termine abbiamo \( k \geq i \) se \( i>j \) allora \( k>j \), da cui la definizione di matrice triangolare superiore.

\subsection{Esercizio 2}
\begin{flushleft}
Una matrice triangolare inferiore $L \in M_{n \times n}$ è detta a diagonale unitaria se i suoi elementi sulla diagonale sono pari a 1:
\[
l_{i,i}=1 \hspace{15pt} \forall i \in [1,..,n] 
\]
Prendiamo una seconda matrice $K \in M_{n \times n}$ triangolare inferiore a diagonale unitaria, calcoliamo il prodotto tra $K$ e $L$:
\[
\sum_{m=1}^n l_{i,m} \cdot k_{m,j} = \underbrace{ \sum_{i<j} l_{i,m} \cdot k_{m,j} }_{0} + \underbrace{ \sum_{i=j} l_{i,m} \cdot k_{m,i} }_{1} + \sum_{i>j} l_{i,m}\cdot k_{m,j}
\]
La risultante matrice assume valori:
\begin{itemize}
\item $\sum_{i<j} l_{i,m} \cdot k_{m,j} = 0 \hspace{15pt} \forall i,j \in [1,..,n] $ 
\item $\sum_{i=j} l_{i,m} \cdot k_{m,j} = 1 \hspace{15pt} \forall i,j \in [1,..,n] $ 
\item $\sum_{i>j} l_{i,m} \cdot k_{m,j} \in  \mathbb{R} \hspace{15pt} \forall i,j \in [1,..,n] $ 
\end{itemize}
che non è altro che la definizione di matrice triangolare inferiore a diagonale unitaria, come volevamo dimostrare.\\

(Allo stesso modo si può dimostrare per matrici triangolari superiori).
\end{flushleft}
\subsection{Esercizio 3}
Sia $A \in M_{n \times n}$ una matrice triangolare superiore non singolare.\\
Se $A$ \'e invertibile allora pu\'o essere scritta come $A=D(\mathit{I_n}+U)$ dove $D$ \'e una matrice diagonale per cui vale $diag(D)=diag(A)$, $\mathit{I_n}$ \'e la matrice identit\'a ed $U$ \'e una matrice triangolare strettamente superiore ($diag(U)=0$).\\
Per le propriet\'a delle matrici triangolari strettamente superiori si ha che $U^n = 0$ (ovvero che la matrice $U$ \'e nilpotente), mentre per le propriet\'a delle matrici diagonali si ha che $D^{-1}$ \'e ancora una matrice diagonale.
\\
Volendo provare che $A^{-1}=(\mathit{I_n}+U)^{-1}D^{-1}$ espandiamo in serie $ (\mathit{I_n}+U)^{-1}$:
\[
(\mathit{I_n}+U)^{-1}=\mathit{I_n}-U+U^2-...+(-1)^{n-1}U^{n-1}.
\]
Essendo questa una serie di somme e prodotti di matrici triangolari superiori la matrice inversa $A^{-1}$ sar\'a in generale una matrice triangolare superiore.


\subsection{Esercizio 4}
\begin{flushleft}
L'eliminazione nella prima colonna richiede $n$ somme ed $n$ prodotti per $n-1$ righe, quindi in totale $(n+n)(n-1) = 2n(n-1)$ \texttt{flops}. L'eliminazione della seconda richiede $n-1$ somme ed $n-1$ prodotti per $n-2$ righe, quindi in totale $[(n-1)+(n-1)](n-2) = 2(n-1)(n-2)$ \texttt{flops}.\\
Procedendo per induzione si ha che il numero totale di operazioni \'e
\[
\sum_{i=0}^{n} 2(n-i)(n-i+1).
\]
Operando la sostituzione $j = n-i+1$ si ha che la somma diviene :
\[
2 \sum_{j=n+1}^{1} j(j-1) = 2\Big( \sum_{j=1}^{n+1}j^2 + \sum_{j=1}^{n+1}j\Big) = 2\Big(\sum_{j=1}^{n-1}j^2 + n^2 + (n+1)^2 + \sum_{j=1}^{n-1}j +n +n+1 \Big) =
\]
\[
= 2\Big( \frac{n\cdot (n-1)\cdot(2n-2)}{6} + \frac{n\cdot(n-1)}{2} + 2n^2 +3n +2 \Big) = 2\Big(\frac{2n^3-4n^2+2n}{6} +\frac{n^2-n}{2} + 2n^2 +3n +2\Big) \leq \frac{2}{3}\cdot n^3
\]
\\

quindi si ha che il numero di flop è $\frac{2}{3}\cdot n^3$, come volevamo dimostrare
\end{flushleft}
\subsection{Esercizio 5}
\lstinputlisting[language=Matlab]{cap_3/LUP.m}

\subsection{Esercizio 6}
\lstinputlisting[language=Matlab]{cap_3/solveLinearLUP.m}

\subsection{Esercizio 7}
\begin{flushleft}
Per essere SDP una matrice $A \in \mathbb{R}^{n \times n}$ deve sottostare a due proprietà:
\begin{itemize}
    \item
    deve essere simmetrica, cioè $A=A^T$;
    \item 
    $\forall x \in \mathbb{R}^n$ tale che $x \neq 0$ vale $x^TAx>0$
\end{itemize}
La matrice $A$ essendo non singolare è anche SDP.
Le matrici $AA^T$ e $A^TA$ per essere SDP devono dimostrare le proprietà sopra:
\begin{itemize}
    \item simmetriche: 
    \[
    (AA^T)^T = (A^T)^TA^T = AA^T
    \]
    \[
    (A^TA)^T = A^T(A^T)^T = A^TA.
    \]
    \item definite positive:
    \[
    x^TAA^Tx = xx^TAA^Txx^T = x(A^Tx)^T(x^TA)^Tx^T = (\underbrace{x^TA^Tx}_{>0})^T \cdot (\underbrace{x^TAx}_{>0})^T > 0
    \]
    \[
    x^TA^TAx = xx^TA^TAxx^T = x(Ax)^T(x^TA^T)^Tx^T = (\underbrace{x^TAx}_{>0})^T \cdot (\underbrace{x^TA^Tx}_{>0})^T > 0
    \]
\end{itemize}
possiamo quindi affemrare che le matrici $AA^T$ e $A^TA$ sono SDP.
\end{flushleft}
\subsection{Esercizio 8}
Se $A \in M_{m \times n}$ con $m \geq n = rank(A)$ allora diremo che $A$ ha rango massimo.
\\
Questo comporta che la matrice sia invertibile, ovvero che il suo determinante $\det(A) \neq 0$. La matrice \'e quindi nonsingolare e di conseguenza simmetrica definita positiva, dalla dimostrazione dell'esercizio \textbf{3.7}.

\subsection{Esercizio 9}
Si ha ovviamente che
\[
A = \frac{1}{2}(A+A^T) + \frac{1}{2}(A-A^T)=\frac{1}{2}A + \frac{1}{2}A^T + \frac{1}{2}A - \frac{1}{2}A^T
\]
\\
Definendo $A_s \doteq \frac{1}{2}(A+A^T)$ si mostra come $A_s = A_s^T$ infatti
\[
\frac{1}{2}(A+A^T) = [ \frac{1}{2}(A+A^T)]^T = \frac{1}{2}(A+A^T)^T = \frac{1}{2}(A^T+(A^T)^T) = \frac{1}{2}(A+A^T)^T.
\]
\\
Da cui $A_s$ \'e detta parte \textbf{simmetrica} di $A$.
\\
Definendo $A_a \doteq \frac{1}{2}(A-A^T)$ si mostra come $A_a = -A_a^T$ infatti
\[
\frac{1}{2}(A-A^T) = -\frac{1}{2}(A-A^T)^T = -\frac{1}{2}(A^T-(A^T)^T) = -\frac{1}{2}(A^T-A) = \frac{1}{2}(A-A^T).
\]
Da cui $A_a$ \'e detta parte \textbf{antisimmetrica} di $A$.
\\
Dato poi un generico vettore $\mathbf{x} \in \mathbf{R}^n$ si ha che $\mathbf{x}^TA\mathbf{x} = \mathbf{x}^TA_s\mathbf{x}$ infatti
\[
\mathbf{x}^TA\mathbf{x} = \mathbf{x}^T(A_s + A_a)\mathbf{x} = \mathbf{x}^TA_s\mathbf{x} + \mathbf{x}^TA_a\mathbf{x} = \frac{1}{2}(\mathbf{x}^TA\mathbf{x} + \mathbf{x}^TA^T\mathbf{x}) + \frac{1}{2}(\mathbf{x}^TA\mathbf{x} - \mathbf{x}^TA^T\mathbf{x}).
\]
Analizzando l'ultimo termine $\frac{1}{2}(\mathbf{x}^TA\mathbf{x} - \mathbf{x}^TA^T\mathbf{x})$ si nota come
\[ \frac{1}{2}(\mathbf{x}^TA\mathbf{x} - \mathbf{x}^TA^T\mathbf{x}) = \frac{1}{2}(\mathbf{x}^TA\mathbf{x} - (A\mathbf{x})^T\mathbf{x})=0
\]
\\
dal momento che, definendo $A\mathbf{x}=\mathbf{y}$ si ha $\mathbf{x}^T\mathbf{y}= \mathbf{y}^T\mathbf{x}.$
Da cui la tesi.  $\qedsymbol$

\subsection{Esercizio 10}
\begin{flushleft}
Prendiamo $i \in [1,...,n]$ colonne della matrice, possiamo vedere che l'algoritmo esegue $i-1$ somme di 2 prodotti quindi $2(i-1)$ e in più esegue un'operazione di sottrazione e una di divisione che equivale a 2 flop. Queste operazioni vengono eseguite per $n-i$ volte, cioè per ogni colonna della matrice, il che significa che il numero di flop sono:
\[
\sum_{i=1}^{n}2(n-i)(i-1) = 2\cdot\sum_{i=1}^{n}(i\cdot n-n-i^2+i) = 2\cdot \Big[(n+1)\sum_{i=1}^{n}i-n^2-\sum_{i=1}^{n}i^2\Big] = 
\]
\[
= 2\cdot \Big[(n+1)\cdot n + \frac{(n+1)(n-1)n}{2} - n^2 - n^2 - \frac{n(n-1)(2n-1)}{6}\Big] = 2\cdot \Big(n - n^2 + \frac{n^3-n}{2} -\frac{2n^3-3n^2+n}{6} \Big) =
\]
\[
= 2\cdot \Big[n^3 \cdot \Big(\frac{1}{2} - \frac{1}{3}\Big) + n^2 \cdot \Big(-1 + \frac{1}{6} + \frac{1}{2}\Big) + n \cdot \Big(1-\frac{1}{2}-\frac{1}{6}\Big) \Big]= \frac{2}{6} n^3 + \frac{2}{3} n^2 + \frac{2}{3} n \approx \frac{1}{3}n^3
\]
quindi l'algoritmo di fattorizzazione $LDL^T$ ha un costo di $\frac{n^3}{3} flop$.
\end{flushleft}
\subsection{Esercizio 11}
\begin{flushleft}
L'algoritmo di fattorizzazione $LDL^T$ da noi implementato è il seguente:
\lstinputlisting[language=matlab]{cap_3/factLDLT.m}
è possibile vedere il funzionamento di questa function nell'esercizio \pageref{es313}.
\end{flushleft}
\subsection{Esercizio 12}
Il codice MatLab corrispondente alla risoluzione di un sistema lineare con matrice di ingresso fattorizzata LDLT è il seguente:
\lstinputlisting[language=matlab]{cap_3/linLDLT.m}
\subsection{Esercizio 13}
\label{es313}
\begin{flushleft}
Per verificarlo abbiamo usato il seguente codice MatLab:
\lstinputlisting[language=Matlab]{cap_3/es13/es13.m}
Che restituisce l'output:
\includegraphics[width=\textwidth]{cap_3/es13/es313.png}
L'output è molto chiaro, la seconda matrice $A_{2}$ non può essere fattorizzata $LDL^T$ di conseguenza non è $SDP$.
\end{flushleft}


\subsection{Esercizio 14}
\label{es314}
\begin{flushleft}
In entrambi i casi abbiamo usato la matrice $A\in M^{3\times 3}$ con elementi:
\[ 
A =
\begin{pmatrix}
    15  & -3 &  2 \\
   -4  &  9  &  2 \\
    6  &  0  &  10 \\
\end{pmatrix}
\]
e il vettore dei termini noti $b\in \mathbb{R}^3$ con valori:
\[
{b} = (3.2, 2.3, 3.1)^T
\]
Usando il codice MatLab sottostante è possibile risolvere questi 2 esempi:
\newpage
\lstinputlisting[language=Matlab]{cap_3/es14/es14.m}
Il codice sopra restituisce l'output:
\begin{figure}[h]
\includegraphics[left, width=400px]{cap_3/es14/es314.png}
\end{figure}
La soluzione ottenuta 
\end{flushleft}
\subsection{Esercizio 15}
\lstinputlisting[language=Matlab]{cap_3/es15/es15.m}

\subsection{Esercizio 16}
\begin{flushleft}
Abbiamo implementato il codice seguente per poter rispondere alle domande dell'esercizio:
\lstinputlisting[language=Matlab]{cap_3/es16/es16.m}
Possiamo confermare che le soluzioni $x$ e $y$ dei sistemi lineari $A\cdot x = b$ e $A\cdot y = c$ sono giuste dato che calcolando i loro residui otteniamo:
\begin{figure}[h]
\includegraphics[left, width=250px]{cap_3/es16/es316}
\end{figure}
\newline \\
Nel passo successivo si usa la serie di istruzioni forniteci dall'esercizio. Nel caso del vettore $x$ si perviene alla stessa soluzione precedentemente fornita dall'esercizio. Invece nel caso del vettore $y$ abbiamo una propagazione degli errori nella soluzione trovata, che si può vedere a partire dall'elemento $y_7$. Possiamo vederlo dall'output:
\begin{figure}[h]
\includegraphics[left, width=250px]{cap_3/es16/es316a}
\end{figure}
\newline \\
Si vede che c'è una perturbazione sulla soluzione che è possibile spiegare andando a studiare la seguente disuguaglianza:
\[
\frac{\left|\left|\Delta x\right|\right|}{\left|\left|x\right|\right|} \leq k(A) \cdot \left(\frac{\left|\left|\Delta c\right|\right|}{\left|\left|c\right|\right|} + \frac{\left|\left|\Delta A\right|\right|}{\left|\left|A\right|\right|}\right) = k(A) \cdot \left(\frac{\left|\left|\Delta c\right|\right|}{\left|\left|0.1\cdot b\right|\right|} + \frac{\left|\left|\Delta A\right|\right|}{\left|\left|A\right|\right|}\right) = 
\]
\[
= k(A) \cdot \left(\frac{\left|\left|\Delta c\right|\right|}{10^{-1}\cdot\left|\left|b\right|\right|} + \frac{\left|\left|\Delta A\right|\right|}{\left|\left|A\right|\right|}\right) = k(A) \cdot \left(10\cdot\frac{\left|\left|\Delta c\right|\right|}{\left|\left|b\right|\right|} + \frac{\left|\left|\Delta A\right|\right|}{\left|\left|A\right|\right|}\right) 
\]
Considerato che:
\begin{itemize}
    \item $\frac{\left|\left|\Delta x\right|\right|}{\left|\left|x\right|\right|}$ può essere assimilato ad una sorta di errore relativo sul risultato
    \item $\frac{\left|\left|\Delta A\right|\right|}{\left|\left|A\right|\right|}$ e $\frac{\left|\left|\Delta c\right|\right|}{\left|\left|c\right|\right|}$ possono essere assimilati ai corrispondenti errori relativi sui dati in ingresso
\end{itemize}
dato che il numero di condizionamento del problema è $k(A)=1.0202\cdot10^{20}$, che è $>>1$ (calcolato nell'esercizio precedente a pag. \pageref{es315}), allora la matrice è mal condizionata.
\end{flushleft}
\newpage
\subsection{Esercizio 17}
\lstinputlisting[language=matlab]{cap_3/factQRH.m}
\subsection{Esercizio 18}
\lstinputlisting[language=Matlab]{cap_3/solveQR.m}

\newpage
\subsection{Esercizio 19}
\lstinputlisting[language=Matlab]{cap_3/es19/es19.m}

\subsection{Esercizio 20}
Risolviamo il sistema di equazioni non lineari applicando il metodo di Newton.
La funzione data è \[F(x_1, x_2)= \begin{cases}x_2-cos(x_1)\\ x_1x_2 -1/2\end{cases}\]
Vogliamo trovare $F(x_1, x_2)=0$ partendo da $x_1(0) = 1\mbox{, } x_2(0) = 1$\\
Troviamo quindi il Jacobiano della funzione: \( J=\begin{pmatrix} sin(x_1) & 1  \\\ x_1 & x_2 \end{pmatrix} \)\\
Applicando il metodo di Newton si va a risolvere:
$\begin{cases} J_F(\underline{x}^{(k)})\underline{d}^{(k)}=-F(\underline{x}^{(k)}) \\ \quad \underline{x}^{(k+1)}=\underline{x}^{(k)}+\underline{d}^{(k)} \end{cases}$\\
Troviamo quindi: $x_1 = 0.6100 \mbox{ e } x_2 =0.8196$

Il codice matlab per il calcolo del minimo è:
\lstinputlisting[language=Matlab]{cap_3/es20/es20.m}

Il codice matlab per la risoluzione di sistemi di equazioni non lineari mediante Newton è:
\lstinputlisting[language=Matlab]{cap_3/NewtonNL.m}

\begin{tabular}{l|c|r}
i & $x_1,x_2$ & norma dell'incremento \\
\hline
1 & 0.7458, 0.7542 & 0.5001 \\
2 & 0.5531, 0.8653 & 0.3145 \\
3 & 0.6042, 0.8241 & 0.0929 \\
4 & 0.6100, 0.8197 & 0.0102 \\
5 & 0.6100, 0.8196 & 0.00013 \\ 


\end{tabular}

\newpage
\subsection{Esercizio 21}
Un punto stazionario $(\hat{x_1}, \hat{x_2})$ è tale per cui $J(\hat{x_1},\hat{x_2})=0$. Si ottiene quindi il sistema non lineare:
$$F(\underline{x})=\underline{0}\mbox{ con }F=\begin{pmatrix}\frac{\partial f}{\partial x_1}\\\frac{\partial f}{\partial x_2}\end{pmatrix}=\begin{pmatrix}4x_1^3+2x_1+x_2\\x_1+2x_2-2\end{pmatrix}.$$
Troviamo il Jacobiano della funzione: \( J=\begin{pmatrix} 12x_1^2+2 & 1  \\\ 1 & 2 \end{pmatrix} \)\\

Troviamo quindi $$\min{f(x_1,x_2)}\approx -0.2573\mbox{ in }(0.4433, -1.2217).$$

Il codice matlab per il calcolo del minimo è:
\lstinputlisting[language=Matlab]{cap_3/es21/es21.m}
\subsection{Funzioni MatLab Usate}
\subsubsection{Algoritmo di fattorizzazione LU con pivoting parziale}
\lstinputlisting[language=matlab]{cap_3/factLUP.m}
\subsubsection{Risoluzione sistema lineare con funzione di ingresso già fattorizzata LU}
\lstinputlisting[language=matlab]{cap_3/triInf.m}
\lstinputlisting[language=matlab]{cap_3/triSup.m}
\subsubsection{Risoluzione diagonale matrice LDLT di un sistema lineare}
\lstinputlisting[language=matlab]{cap_3/linDiag.m}