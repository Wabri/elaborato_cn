\begin{flushleft}
Una matrice $L\in M_{n\times n}$ è definita triangolare inferiore se preso $l_{i,j} \in L$ vale la proprietà: 
\[
l_{i,j}=0 \hspace{15pt} \forall i<j \hspace{10pt} i,j \in [1,..,n] 
\]
Possiamo dimostrare facilmente che la somma di due matrici triangolari inferiori è ancora una matrice triangolare inferiore. Prendiamo due matrici $L,K \in M_{n \times n}$ per definizione di triangolare inferiore si deve valere che:
\[
l_{i,j} + k_{i,j} = 0 + 0 = 0 \hspace{15pt} \forall i<j \hspace{10pt} i,j \in [1,..,n] 
\]
che è la definizione di matrice triangolare inferiore, come volevasi dimostrare. \\

Dimostriamo ora che il prodotto di due matrici triangolari inferiori è ancora una matrice triangolare inferiore.
Indichiamo con $A \in M_{n \times n}$ la matrice risultante del prodotto delle 2 matrici $L$ e $K$, gli elementi della nuova matrice, $a_{i,j} \in A$, sono calcolati come la somma del prodotto degli elementi delle due matrici:
\[
a_{i,j} = \sum_{m=1}^n l_{i,m} \cdot k_{m,j} \hspace{35pt} \forall i,j \in [1,..,n].
\]
questa somma può essere scritta anche:
\[
\sum_{m=1}^n l_{i,m} \cdot k_{m,j} =  \underbrace{ \sum_{i<j} l_{i,m} \cdot k_{m,j} }_{0}  + \sum_{i\geq j} l_{i,m}\cdot k_{m,j}
\]
da quest'ultima il valore $0 =\sum_{i<j} l_{i,m} \cdot k_{m,j} = a_{i,j} \hspace{10pt} i,j \in [1,..,n]$ che è la definizione di matrice triangolare inferiore, come volevasi dimostrare. \\

(Allo stesso modo si può dimostrare per matrici triangolari superiori).
\end{flushleft}