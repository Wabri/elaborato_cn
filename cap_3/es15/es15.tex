\begin{flushleft}
\lstinputlisting[language=Matlab]{cap_3/es15/es15.m}
Il codice MatLab sopra restituisce in output:
\begin{lstlisting}[language=matlab, basicstyle = \small]
ans =
    101.000000000000e+000
ans =
    101.000000000000e+000
Warning: Matrix is close to singular or badly scaled.
Results may be inaccurate. RCOND =  9.801980e-21. 
> In cond (line 46)
  In es15 (line 7) 
ans =
    102.020202020202e+018
\end{lstlisting}
Analiticamente abbiamo che $\rVert A \rVert_{\infty} = \rVert A \rVert_1 = 101$, come \'e possibile verificare attraverso l'istruzione norm di Matlab.
Per quanto riguarda il calcolo del numero di condizionamento $k_{\infty}(A)$ abbiamo che, andando a calcolare l'inversa, gli elementi della matrice crescono di 2 ordini di grandezza per ogni riga, arrivando ad avere valori prossimi a $10^{18}$. Questo implica che $\rVert A^{-1} \rVert_{\infty}>10^{20}$ per cui possiamo affermare che il problema \'e malcondizionato.
La verifica con l'istruzione cond di Matlab restituisce un warning in cui avverte che $RCOND=9.801980e-21$, confermando quanto appena detto.
\end{flushleft}