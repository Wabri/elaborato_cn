\begin{flushleft}
La funzione data è \[F(x_1, x_2)= \begin{cases}x_2-cos(x_1)\\ x_1x_2 -1/2\end{cases}\]
Vogliamo trovare $F(x_1, x_2)=0$ partendo da $x_1(0) = 1\mbox{, } x_2(0) = 1$\\
Troviamo quindi il Jacobiano della funzione: 
\[ 
J=\begin{pmatrix} sin(x_1) & 1  \\\ x_1 & x_2 \end{pmatrix}
\]
Applicando il metodo di Newton si va a risolvere:
\[
\begin{cases} 
J_F(\underline{x}^{(k)})\underline{d}^{(k)}=-F(\underline{x}^{(k)}) \\ 
\quad \underline{x}^{(k+1)}=\underline{x}^{(k)}+\underline{d}^{(k)} 
\end{cases}
\]
Usando il codice MatLab dell'esercizio successivo (\texttt{Esercizio 21}), si ottengono i risultati sotto:
\[
 x_1 = 0.6100 \mbox{ e } x_2 =0.8196
\]
con la rispettiva tabella:
\begin{center}
\begin{tabular}{l|c|c|c}
i & $x_1$ & $x_2$ & Norma incremento \\
\hline
1 & 0.7458 & 0.7542 & 0.5001 \\
2 & 0.5531 & 0.8653 & 0.3145 \\
3 & 0.6042 & 0.8241 & 0.0929 \\
4 & 0.6100 & 0.8197 & 0.0102 \\
5 & 0.6100 & 0.8196 & 0.00013 \\ 
\end{tabular}
\end{center}
\end{flushleft}