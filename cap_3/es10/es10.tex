\begin{flushleft}
Prendiamo $i \in [1,...,n]$ colonne della matrice, possiamo vedere che l'algoritmo esegue $i-1$ somme di 2 prodotti quindi $2(i-1)$ e in più esegue un'operazione di sottrazione e una di divisione che equivale a 2 flop. Queste operazioni vengono eseguite per $n-i$ volte, cioè per ogni colonna della matrice, il che significa che il numero di flop sono:
\[
\sum_{i=1}^{n}2(n-i)(i-1) = 2\cdot\sum_{i=1}^{n}(i\cdot n-n-i^2+i) = 2\cdot \Big[(n+1)\sum_{i=1}^{n}i-n^2-\sum_{i=1}^{n}i^2\Big] = 
\]
\[
= 2\cdot \Big[(n+1)\cdot n + \frac{(n+1)(n-1)n}{2} - n^2 - n^2 - \frac{n(n-1)(2n-1)}{6}\Big] = 2\cdot \Big(n - n^2 + \frac{n^3-n}{2} -\frac{2n^3-3n^2+n}{6} \Big) =
\]
\[
= 2\cdot \Big[n^3 \cdot \Big(\frac{1}{2} - \frac{1}{3}\Big) + n^2 \cdot \Big(-1 + \frac{1}{6} + \frac{1}{2}\Big) + n \cdot \Big(1-\frac{1}{2}-\frac{1}{6}\Big) \Big]= \frac{2}{6} n^3 + \frac{2}{3} n^2 + \frac{2}{3} n \approx \frac{1}{3}n^3
\]
quindi l'algoritmo di fattorizzazione $LDL^T$ ha un costo di $\frac{n^3}{3} flop$.
\end{flushleft}