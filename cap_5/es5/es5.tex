\lstinputlisting[language=Matlab]{cap_5/es5/es5.m}

\begin{flushleft}
Il sistema $Ax=b$ è formato dagli elementi:
\[
A = \left(\begin{array}{ccc} -4 & 2 & 1\\ 1 & 6 & 2\\ 1 & -2 & 5 \end{array}\right)
\]
\[
b= \left(\begin{array}{c} 1\\ 2\\ 3 \end{array}\right)
\]
partendo dal vettore iniziale 
\[
x_0 = \left(\begin{array}{c} 0\\ 0\\ 0 \end{array}\right)
\]
il metodo di Jacobi mi restituisce in output il vettore:
\[
\left(\begin{array}{c} -0.02682\\ 0.1201\\ 0.6534 \end{array}\right)
\] 
con il metodo di Gauss-Seidel si ottiene invece il vettore:
\[
\left(\begin{array}{c} -0.02688\\ 0.12\\ 0.6532 \end{array}\right)
\]

Oltre a calcolare il vettore risultante, le due funzioni usate nel codice mi dicono anche quante iterazioni avvengono:
\begin{center}
\begin{tabular}{c|c}
\hline
Metodo & Iterazioni\\
\hline
Jacobi & 12 \\
Gauss-Seidel & 8\\
\hline
\end{tabular}
\end{center}
\end{flushleft}