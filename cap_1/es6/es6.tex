Il codice MatLab, indicando con x=$x_n$ e r=$\epsilon$:
\lstinputlisting[language=Matlab]{cap_1/es6/es6.m}
\newpage
restituisce i valori:
\begin{center}
\begin{tabular}{c|c|c}
n & $x_n$ & $\epsilon$ \\
\hline
    0 & 2.00000000000000e+000 & 585.786437626905e-003\\
    1 & 1.50000000000000e+000 & 85.7864376269049e-003\\
    2 & 1.42857142857143e+000 & 14.3578661983335e-003\\
    3 & 1.41463414634146e+000 & 420.583968367971e-006\\
    4 & 1.41421568627451e+000 & 2.12390141496321e-006\\
    5 & 1.41421356268887e+000 & 315.774073555986e-012\\
    6 & 1.41421356237310e+000 & 0.00000000000000e+000\\
    7 & 1.41421356237310e+000 & 0.00000000000000e+000\\
\end{tabular}
\end{center}
I valori indicano che per valori di $n$ superiori a 5 l'errore, indicato con $\epsilon$, è dell'ordine di \(10^{-12}\).