L'espressione aritmetica dei due algoritmi è:\\
\begin{itemize}
\item{1)} (x+y)+z = x+y+z 
\item{2)}  x+(y+z) = x+y+z 
\end{itemize}
Ne consegue che sono equivalenti
In aritmetica finita, invece, avremo:
\begin{itemize}
\item{1)} \( (x\oplus y)\oplus z = fl(fl(fl(x) + fl(y)) +fl(z)) = \)\\
\( =((x(1+\varepsilon_x) + y(1+\varepsilon_y))(1+\varepsilon_A) +z(1+\varepsilon_z))(1+\varepsilon_B) = \) \\
\[\varepsilon_R =\frac{(x(1+\varepsilon_x)(1+\varepsilon_A)(1+\varepsilon_B) + y(1+\varepsilon_y)(1+\varepsilon_A)(1+\varepsilon_B)+z(1+\varepsilon_z))(1+\varepsilon_B) -x-y-z}{z+y+z}     \]
prendo \( \varepsilon_M = max\{\varepsilon_x, \varepsilon_y, \varepsilon_z, \varepsilon_A, \varepsilon_B\}\)
\[\varepsilon_R \leq \frac{x(1+\varepsilon_M)^3 + y(1+\varepsilon_M)^3+z(1+\varepsilon_M)^2 -x-y-z}{z+y+z} = \]
\[ =  \frac{x(3\varepsilon_M + 3\varepsilon_M^2 +\varepsilon_M^3) + y(3\varepsilon_M + 3\varepsilon_M^2 +\varepsilon_M^3)+z(2\varepsilon_M + \varepsilon_M^2)}{z+y+z}    \]
Sapendo che \(\varepsilon_M \leq 1 \) posso affermare che \( \varepsilon_M \geq \varepsilon_M^2 \geq \varepsilon_M?3 \) ed effettuare un'altra minorazione

\[ \varepsilon_R \leq \frac{7x\varepsilon_M + 7y\varepsilon_M + 3z\varepsilon_M}{x+y+z} = \varepsilon_M(3+4\frac{x+y}{x+y+z}) \]

\item{2)}  \( x\oplus (y\oplus z) =fl(fl(x) +fl( fl(y) +fl(z)))     \) \\
Il procedimento sarà analogo a quello visto prima con lo scambio degli addendi al nominatore della frazione
\[ \varepsilon_R \leq \frac{7z\varepsilon_M + 7y\varepsilon_M + 3x\varepsilon_M}{x+y+z} = \varepsilon_M(3+4\frac{z+y}{x+y+z}) \]
\end{itemize}