Le due espressioni in aritmetica finita vengono scritte tenendo conto dell'errore di approssimazione sul valore reale:
\begin{itemize}
\item $fl(fl(fl(x)+fl(y))+fl(z)) = \\ = ((x(1+\varepsilon_{x})+y(1+\varepsilon_{y}))(1+\varepsilon_{a})+z(1+\varepsilon_{z}))(1+\varepsilon_{b})$
\item $fl(fl(x)+fl(fl(y)+fl(z))) = \\ = (x(1+\varepsilon_{x})+(y(1+\varepsilon_{y})+z(1+\varepsilon_{z}))(1+\varepsilon_{a}))(1+\varepsilon_{b})$
\end{itemize}
Indicando con $\varepsilon_{x},\varepsilon_{y},\varepsilon_{z}$ i relativi errori di $x, y, z$ e con $\varepsilon_{a},\varepsilon_{b}$ gli errori delle somme. \newline
Per calcolare l'errore relativo delle due espressioni consideriamo $\varepsilon_{m} = max\{\varepsilon_{x},\varepsilon_{y},\varepsilon_{z},\varepsilon_{a},\varepsilon_{b}\}$, dalla definizione di errore relativo si ha quindi:
\begin{itemize}
    \item 
    \[ 
    \varepsilon_{1} = \frac{((x(1+\varepsilon_{x})+y(1+\varepsilon_{y}))(1+\varepsilon_{a})+z(1+\varepsilon_{z}))(1+\varepsilon_{b})-(x+y+z)}{x+y+z} \approx
    \]
    \[
    \approx \frac{x(1+\varepsilon_{x}+\varepsilon_{a}+\varepsilon_{b})+y(1+\varepsilon_{y}+\varepsilon_{a}+\varepsilon_{b})+z(1+\varepsilon_{z}+\varepsilon_{b})-x-y-z}{x+y+z} \leq
    \]
    \[ 
    \leq \frac{3\cdot x\cdot\varepsilon_{m}+ 3\cdot y\cdot\varepsilon_{m} + 2\cdot z\cdot\varepsilon_{m}}{x+y+z} \leq \frac{3\cdot\varepsilon_{m}\cdot(x+y+z)}{x+y+z} = 3\cdot\varepsilon_{m}
    \]
    \item seguendo gli stessi procedimenti del punto sopra possiamo scrivere:
    \[
    \varepsilon_{2} = \frac{(x(1+\varepsilon_{x})+(y(1+\varepsilon_{y})+z(1+\varepsilon_{z}))(1+\varepsilon_{a}))(1+\varepsilon_{b})-(x+y+z)}{x+y+z} = 
    \]
    \[
    = ... \leq \frac{2\cdot x\cdot\varepsilon_{m}+ 3\cdot y\cdot\varepsilon_{m} + 3\cdot z\cdot\varepsilon_{m}}{x+y+z} \leq \frac{3\cdot\varepsilon_{m}\cdot(x+y+z)}{x+y+z} = 3\cdot\varepsilon_{m}
    \]
\end{itemize}