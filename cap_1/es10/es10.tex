All'interno della radice può presentarsi un problema di overflow dato che la somma dei due quadrati potrebbe essere molto grande, tanto grande da poter superare il limite massimo rappresentabile dalla macchina:
\[
realmax = (1-b^{-m})\cdot b^{b^{s}-\nu}
\]
Per risolvere questo problema è necessario prendere il massimo valore tra le due variabili:
\[
m = max\{|x|,|y|\}
\]
e moltiplicare e dividere per questo valore:
\[
\sqrt{x^2 + y^2} = m\cdot\frac{\sqrt{x^2+y^2}}{m} = m\cdot\sqrt{\frac{x^2+y^2}{m^2}} = m\cdot\sqrt{\biggl(\frac{x}{m}\biggr)^2+\biggl(\frac{y}{m}\biggr)^2}
\]
In questo modo si eviterà il problema di overflow, il problema è ben condizionato dato che potenza e radice sono ben condizionate e grazie alla modifica proposta indicata sopra.