Per dimostrare le due uguaglianze è necessario sviluppare in serie di taylor $f(x)$ fino al secondo ordine:
\[
f(x) = f(x_0) + (x-x_0)f'(x_0)+\frac{(x-x_0)^2f''(x_0)}{2} + O((x-x_0)^2)
\]
Da cui possiamo sostituire con i valori di $x=x+h$ e $x=x-h$:
\[
f(x_0+h)=f(x_0)+h f'(x_0)+\frac{h^2f''(x_0)}{2} + O(h^2)
\]
\[
f(x_0-h)=f(x_0)-h f'(x_0)+\frac{h^2f''(x_0)}{2} + O(h^2)
\]
Andando a sostituire questi valori si ottiene, nel primo caso:
\[
\frac{f(x_0 + h) - f(x_0 + h)}{2h} =
\]
\[
= \frac{(f(x_0)+h f'(x_0)+\frac{h^2f''(x_0)}{2} + O(h^2))-(f(x_0)-h f'(x_0)+\frac{h^2f''(x_0)}{2} + O(h^2))}{2h} = 
\]
\[
= \frac{2h f'(x_0)+O(h^2)}{2h} = f'(x_0)+O(h^2)
\]
nel secondo caso:
\[
\frac{f(x_0 + h) -2f(x_0) - f(x_0 + h)}{h^2} =
\]
\[
= \frac{f(x_0)+h f'(x_0)+\frac{h^2f''(x_0)}{2} + O(h^2) -2f(x_0) + f(x_0)-h f'(x_0)+\frac{h^2f''(x_0)}{2} + O(h^2)}{h^2} = 
\]
\[
= \frac{h^2f''(x_0)+O(h^2)}{h^2} = f''(x_0)+O(h^2)
\]
\
Abbiamo quindi dimostrato che:
\[
\frac{f(x_0 + h) - f(x_0 + h)}{2h} = f'(x_0)+O(h^2)
\]
\[
\frac{f(x_0 + h) -2f(x_0) - f(x_0 + h)}{h^2} = f''(x_0)+O(h^2)
\]