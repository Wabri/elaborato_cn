Tesi:\\
Sia f(x) una funzione sufficentemente regolare e h>0 una quantità ``piccola''\\
Ipotesi:\\
\[
\frac{f(x_0 + h) - f(x_0 + h)}{2h} = f'(x_0) + O(h^2)
\]
\[
\frac{f(x_0 + h) -2f(x_0) - f(x_0 + h)}{h^2} = f''(x_0) + O(h^2)
\]
Dimostrazione:\\
Sviluppiamo la funzione f(x) mediante il polinomio di taylor al secondo ordine\\
\[
f(x) = f(x_0) + (x-x_0)f'(x_0)+\frac{(x-x_0)^2}{2}f''(x_0) + O((x-x_0)^3)
\]
Sostituiamo \[x=(x_0 +h)\] e  \[x=(x_0-h)\]
\[
f((x_0 +h) = f(x_0) + hf'(x_0)+\frac{h^2}{2}f''(x_0) + O(h^3)
\]
\[
f((x_0 -h) = f(x_0) - hf'(x_0)+\frac{h^2}{2}f''(x_0) + O(h^3)
\]
Risostituendo  nel rapporto incrementale dell'ipotesi otteniamo:
\[
\frac{ f(x_0) + hf'(x_0)+\frac{h^2}{2}f''(x_0) + O(h^3) - f(x_0) - hf'(x_0)+\frac{h^2}{2}f''(x_0) + O(h^3)}{2h} = \frac{2hf'(x_0) + O(h^3)}{2h} =f'(x_0) + O(h^2) \qedsymbol
\]

Per la seconda uguaglianza dell'ipotesi basterà applicare lo stesso procedimento con uno sviluppo di taylor al 3° ordine.
