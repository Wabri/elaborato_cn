Sapendo che il metodo iterativo è convergente a \( x^* \) allora per definizione si ha:
 \[
	\lim_{k \to +\infty}\ x_k = x^*
\]
 inoltre per definizione di \( \Phi \) si calcola il limite:
\[
    \lim_{k \to +\infty} \Phi(x_k) = \lim_{k \to +\infty} x_{k+1} = x^*
\]
 infine ipotizzando che la funzione \( \Phi \) sia uniformemente continua, è possibile calcolare il limite:
\[
    \lim_{k \to +\infty} \Phi(x_k) = \Phi(\lim_{k \to +\infty} x_k) = \Phi(x^*)
\]
 dai due limiti si ha la tesi:
\[
    \Phi(x^*) = x^* 
\]