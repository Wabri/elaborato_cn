Sapendo che la rappresentazione del numero è stata fatta usando l'arrotondamento, la precisione di macchina si calcola:
\[
u = \frac{b^{1-m}}{2}
\]
il cui valore sappiamo essere pari a:
\[
u \approx 4.66 \cdot 10^{-10}
\]
dato che stiamo cercando il numero di cifre binarie allora si deve avere $b=2$, è quindi possibile ricavare $m$:
\[
m = 1- log_2{(2\cdot 4.66 \cdot 10 ^{-10})} = 1 - log_2{(9.32 \cdot 10^{-10})} = 1 - (-29.9999999) \approx 31
\]
possiamo pertanto affermare che servono 31 cifre dedicate alla mantissa per rappresentare il numero con precisione macchina \(4.66 \cdot 10^{-10}\).