Sapendo che la rappresentazione del numero è stata fatta usando l'arrotondamento allora la precisione di macchina si calcola:
\[
u = \frac{b^{1-m}}{2}
\]
che sappiamo essere pari a:
\[
u \approx 4.66 \cdot 10^{-10}
\]
dato che stiamo cercando il numero di cifre binarie allora si deve avere $b=2$, è possibile quindi ricavarsi $m$:
\[
m = 1- log_2{(4.66 \cdot 10 ^{-10})} \approx 31.99
\]
possiamo quindi affermare che servono 32 cifre dedicate alla mantissa per rappresentare il numero con precisione macchina \(4.66 \cdot 10^{-10}\).