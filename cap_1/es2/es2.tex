Dal momento che le variabili intere di 2 byte in Fortran vengono gestite in Modulo e Segno, la variabile \texttt{numero} inizializzata con:
\begin{verbatim}
integer*2 numero
\end{verbatim}
varia tra \( -32768 \leq \texttt{numero} \leq 32767 \) (\( - 2^{15} \leq \texttt{numero} \leq 2^{15} - 1 \)). \\ 
Durante la terza iterazione del primo ciclo for si arriva al valore massimo rappresentabile tramite gli interi a 2 byte; alla quarta iterazione si avrà quindi la somma del \texttt{numero} in modulo e segno:
\[
(32767)_{10}+(1)_{10} = (0111111111111111)_{2,MS} + (0000000000000001)_{2,MS} = 
\] \( 
= (1000000000000000)_{2,MS} = (-32768)_{10} 
\) \\ \\
Nel secondo ciclo for, durante la quinta iterazione, al \texttt{numero} viene sottratto 1:
\[
(-32768)_{10}-(1)_{10} = (1000000000000000)_{2,MS} - (0000000000000001)_{2,MS} = 
\] \( 
= (0111111111111111)_{2,MS} = (32767)_{10} 
\) \\ \\
Da cui si spiega l'output del codice.