Il seguente codice
%\lstinputlisting{cap_1/es13/es13.m}
plotta la funzione\ref{es113}
Il  motivo per cui $f(x)$ in $x=\frac{4}{3}$ \'e pari ad un valore finito risiede nel fatto che $\frac{4}{3}$ \'e un numero razionale periodico.
Il calcolatore memorizzerà $fl(\frac{4}{3})$ con un approssimazione determinata dalla precisione macchina in uso.
Calcolando infatti l'espressione mostrata a riga 12 del listato abbiamo questo output:\\   2.220446049250313e-16\\
Questo valore analiticamente dovrebbe essere 0, ma dato che la memorizzazione di $\frac{4}{3}$ non \'e precisa, il valore della funzione nel punto sarà un numero finito.
