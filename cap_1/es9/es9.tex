\lstinputlisting[language=Matlab]{cap_1/es9/es9.m}
dato che il valore di $delta=[0,1]_{10}$ in binario si scrive $delta=[0,\overline{00011}]_2$ allora si nota che la rappresentazione del valore di delta in binario è periodica. Al passo 10 la rappresentazione di $x$ sarà diversa da 1, perchè somma di numeri periodici, essendo $x~=1$ l'unica condizione di uscita dello while il ciclo non si arresterà mai. Possiamo provarlo effettuando le somme binarie:
\[
\Big[\frac{1}{10}\Big]_{10}=\Big[0,\overline{00011}\Big]_2
\]
\[
\Big[0,\overline{00011}\Big]_2+\Big[0,\overline{00011}\Big]_2+ \underbrace{...}_{6volte}+\Big[0,\overline{00011}\Big]_2+\Big[0,\overline{00011}\Big]_2 = 
\]
\[
= [1,00010]_2 \approx [1.0625]_{10} \neq [1.0000]_{10}
\]

che spiegherebbe il motivo del loop dello while.